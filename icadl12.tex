\documentclass[runningheads,a4paper]{llncs}

\usepackage{amssymb}
\setcounter{tocdepth}{3}
\usepackage{graphicx}

\usepackage{url}

\begin{document}

\mainmatter

\title{Automatic Perspective Correction of Manuscript Images}

\author{Ryan Baumann\inst{1} \and Christopher Blackwell\inst{2} \and W. Brent Seales\inst{1}}

\institute{University of Kentucky
\and
Furman University}

\maketitle

\section{Introduction}

\section{Related Work}

Numerous work exists for performing document perspective correction by exploiting the presence of
common text features, such as linear text baselines in printed texts.
As a result, many of these techniques need to be tweaked depending upon morphological features of
the written language of the document in question.
However, for our documents,
many pages had no such features, consisting only of handwritten text and physical plant
specimens. As a result, our approach is fairly unique in that it uses only physical page boundaries to
perform the perspective correction. An advantage to this is that it can handle “blank” pages or pages
consisting only of images or haphazard writing. In addition, because the manuscript pages we are dealing with have
cockled, uneven edges, our boundary detection needs to be relatively robust to the “noise” of these uneven edges.

\section{Conclusion}

\end{document}
